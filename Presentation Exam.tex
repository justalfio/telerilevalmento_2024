\documentclass{beamer} % classe di documenti per poter creare le presentazioni che in questo linguaggio vengono definite come frame. 
\usepackage{graphicx} % Package necessario per utilizzare immagini
\usepackage{listings} % Pacchetto che può essere utilizzato per caricare degli algoritmi direttamente dagli script.
\usetheme{CambridgeUS}
\usecolortheme{orchid}

\logo{\includegraphics[width=1cm]{logounibo.png}} 

\title{Deterioration of Sundarban mangrove forest ecosystem}
\author{Alfio Tomarchio}
\date{19 Giugno 2024}

\begin{document}

\maketitle

\AtBeginSection[]
{
\begin{frame}
\frametitle{Outline}    
    \tableofcontents[currentsection]
\end{frame}
}

\section{Introduzione}

        \begin{frame}{Le mangrovie Sundarbans}
            \begin{itemize}
                \item Le Sundarbans sono la più grande foresta di mangrovie del mondo, situate tra India e Bangladesh
                \item Riduzione della loro estensione a causa dell'inquinamento, sfruttamento e fattori ecologici stocastici
                \item Monitoraggio dell'ecosistema in Mongla
\end{itemize}
\begin{columns}
\begin{column}{0.5\textwidth}
\begin{center}
\caption{Novembre 2016}
\bigskip
  \includegraphics[width=0.7\textwidth]{mongla2016.jpeg}
    \end{center}
\end{column}
\begin{column}{0.5\textwidth}  
\begin{center}
\caption{Novembre 2021}
\bigskip
\includegraphics[width=0.7\textwidth]{mongla2021.jpeg}
\end{center}
\end{column}
\end{columns}

  \end{frame} 


  
        \begin{frame}{Raccolta dati}
Sentinel 2 - Copernicus Open Hub \includegraphics[width=\textwidth]{}
\begin{wrapfigure}{r}{0.25\textwidth} %this figure will be at the right
    \centering
    \includegraphics[width=0.30\textwidth]{copernicus.png}
\end{wrapfigure}
\bigskip
\begin{itemize}
    \item Individuare l'area di ricerca
    \item Scegliere una bassa copertura nuvolosa 
    \item Scegliere le date di confronto
    \item Selezionare True color come livello
    \item Download immagine in formato tiff 16 bit con le 4 bande (2,3,4,8)
\end{itemize}
 \end{frame}

        \begin{frame}{Pacchetti utilizzati}
Per la realizzazione di questo progetto sono stati utilizzati questi pacchetti: 
\bigskip
\begin{itemize}
    \item    \texttt{library(terra)} 
    \item    \texttt{library(imageRy)} 
    \item    \texttt{library(ggplot2)} 
    \item    \texttt{library(viridis)}
    \item    \texttt{library(patchwork)}
    \end{itemize}
        \end{frame}

\section{Indici e funzioni}
        \begin{frame}{DVI ED NDVI}
    I seguenti indici sono stati usati per calcolare la variazione delle mangrovie in Mongla:
    \begin{itemize} 
    \item \textbf{Difference Vegetation Index}:\begin{equation}
DVI={NIR-RED}
    \end{equation}
    \item \textbf{Normalized DIfference Vegetation Index:} \begin{equation}
NDVI=\frac{NIR-RED}{NIR+RED}
    \end{equation}
\end{itemize}
\bigskip
\bigskip
Confronto dati forniti dagli indici delle annate 2016 e 2021, concentrandosi sul mese di novembre data la minore copertura nuvolosa
        \end{frame}
        
        \begin{frame}{Applicazioni packages e R}
        Sono state usate le seguenti funzioni di R e dei pacchetti in esso installati:
        \bigskip
\begin{columns}
\begin{column}{0.5\textwidth}
\begin{itemize}
    \item    \texttt{setwd()} 
    \item    \texttt{rast} 
    \item    \texttt{concatenate} 
    \item    \texttt{im.plotRGB}
    \item    \texttt{par}
    \item    \texttt{dev.off}
    \item    \texttt{colorRampPalette}
\end{itemize}
\end{column}
\begin{column}{0.5\textwidth}  
 \begin{itemize}
    \item    \texttt{plot}
    \item    \texttt{im.classify}
    \item    \texttt{freq}
    \item    \texttt{R come calcolatrice}
    \item    \texttt{data.frame}
    \item    \texttt{ggplot}
    \item    \texttt{im.pca}
    \item    \texttt{focal}
    \end{itemize}
\end{column}
\end{columns}
\end{frame}
        
        \begin{frame}{PlotRGB con NIR}
Effettuo un primo confronto dei due anni posizionando il NIR sul verde, in modo che la vegetazione sia risaltata con quest'ultimo colore:
\begin{figure}
        \centering
        \includegraphics[width=0.8\linewidth]{plotRGB 16 21.jpeg}
        \caption{Plot dei due stack}
    \end{figure}

\end{frame}
        


\section{Risultati NDVI}

        \begin{frame}{Risultati NDVI}
Plot che mostra i risultati dell'indice NDVI, utilizzando la colorazione "viridis" dove il blu scuro indica ciò che non è vegetazione, mentre ciò che va a schiarirsi è rappresentata dalle mangrovie:
\bigskip
\begin{columns}
    \begin{column}{0.6\textwidth}
    \begin{center}
        \includegraphics[width=\textwidth]{NDVI20216.jpeg}
        \caption{NDVI 2016}
        \end{center}
\end{column}
\begin{column}{0.6\textwidth}
\begin{center}
\includegraphics[width=\textwidth]{NDVI 2021.jpeg}
\caption{NDVI 2021}
\end{center}
\end{column}
\end{columns}
     \end{frame}

\section{Risultati classificazione}
\begin{frame}{Risultati classificazione}
Quantifichiamo il cambiamento con \texttt{im.classify}:
\bigskip
\begin{columns}
\begin{column}{0.5\textwidth}
\begin{figure}
    \centering
    \includegraphics[width=1\linewidth]{im.classify2016.png}
\end{figure}
\end{column}
\begin{column}{0.5\textwidth}
\begin{figure}
    \centering
    \includegraphics[width=1\linewidth]{im.classify2021.png}
\end{figure}
\end{column}
\end{columns}
    
\end{frame}
            %\lstinputlisting[language=R]{} % si specifica il linguaggio di programmazione utilizzato all'interno dello script e successivamente i passa come argomento della funzione il nome dello script che dovrà essere caricato su overleaf tramite il tasto carica
            \begin{frame}{Frequenza e totale dei pixel}
            Calcolata la frequenza dei pixel per ogni classe e il totale dei pixel per ogni immagine
ne otteniamo poi la percentuale osservando che:
\bigskip
\bigskip
\bigskip
\begin{columns}
\begin{column}{0.5\textwidth}
\textbf{Novembre 2016:}
\begin{itemize}
    \item Non vegetation: 37,99\%
    \item Mangroves: 62,01\%
\end{itemize}
\end{column}
\begin{column}{0.5\textwidth}  
\textbf{Novembre 2021:}
 \begin{itemize}
    \item Non vegetation: 39,22\%
    \item Mangroves: 60,70\%
    \end{itemize}

            \end{column}
            \end{columns}
            \end{frame}

            \begin{frame}{Grafici}
            Visualizzazione patchwork dei due ggplot in base ai risultati della classificazione:
            \begin{figure}
                \centering
                \includegraphics[width=1\linewidth]{patcwhork dei due ggplot.jpeg}
            \end{figure}
                
            \end{frame}

\section{PCA E Deviazione standard}
\begin{frame}{PCA 2016}
Svolta l'analisi delle componenti principali e la PCA ci permette poi di poter scegliere la PC1 (più rappresentativa) per il calcolo della deviazione standard:
    \begin{SCfigure}
   \centering
\includegraphics[width=1\textwidth]{PCA 2016.jpeg}
 \caption{tot pc1= 77,1\%
tot pc2= 18\%
tot pc3= 3,1\%
tot pc4= 1,7\%}
    
    \end{SCfigure}  
     \end{frame}

    \begin{frame}{PCA 2021}
    Analisi PC1 della PCA 2021:
    \begin{SCfigure}
            \centering
            \includegraphics[width=1\linewidth]{PCA 2021.jpeg}
            \caption{tot pc1= 75,8\%
tot pc2= 20,3\%
tot pc3= 2,5\%
tot pc4= 1,4\%}
            
    \end{SCfigure}
        
    \end{frame}

    \begin{frame}{Moving window}
    Con la tecnica della moving window, in particolare una finestra 3x3 pixel, calcoliamo la deviazione standard sulla PC1:
     \bigskip
     \begin{columns}
     \begin{column}{0.6\textwidth}
\begin{center}
    \includegraphics[width=\textwidth]{Deviazione standard 2016.jpeg}
    \caption{2016}
\end{center}
\end{column}
\begin{column}{0.6\textwidth}  
    \begin{center}
     \includegraphics[width=\textwidth]{deviazione standard 2021.jpeg}
     \caption{2021}
     \end{center}
\end{column}
\end{columns}
 \end{frame}

                
\section{Conclusioni}

        \begin{frame}{Conclusioni}
            \begin{itemize}
                \item 
                Durante la sua storia, la foresta di Sundarbans è sempre stata soggetta a gestione e sfruttamento antropico;
                \pause \item 
                La combinazione di inquinamento, sfruttamento agricolo e rischi ciclonici (dovuti al cambiamento climatico) hanno costantemente reso problematica la conservazione di tale ecosistema;
                \pause \item
                L'obiettivo del progetto è stato quello di evidenziare  la variazione nella copertura delle mangrovie in prossimità di centri urbani, Mongla in questo caso;  
                \pause \item Il risultato ottenuto dallo studio in Mongla ha rispecchiato l'andamento costante dal 2016 al 2021 della foresta di Sundarbans, ossia una riduzione annuale dell'estensione delle mangrovie dello 0,4\% e di 7.7 \mathbf{km^2}.
             
        
            \end{itemize}
            
     \end{frame}
\setbeamertemplate{background}{\includegraphics[width=\paperwidth, height=\paperwidth, keepaspectratio]{Sundarban-National-Park.jpg}}

\begin{frame} 
   \bigskip
   \bigskip
   \bigskip
   \bigskip
   \bigskip
   \bigskip
   \bigskip
   \bigskip
   \bigskip
   \bigskip
   \bigskip
   \bigskip
   \bigskip
\textcolor{white}{GRAZIE PER L'ATTENZIONE!}
   \centering
   \textcolor{white}{\url{https://github.com/justalfio}}
      \centering


\end{frame}

\end{document}
